\documentclass{article}
\usepackage{amsmath}
\usepackage{amssymb}
\title{CS-521-900, Assignment 1}
\author{Yiyun Zhang}
\begin{document}
\maketitle
$ $
\\
1.
\\
(a).
\\
When $n=0$, Left side: $$\sum_{i=0}^{0}i(i+1)=0\cdot(0+1)= 0$$
Right side: $$\dfrac{1}{6}n(n+1)(2n+4)=\dfrac{1}{6}\cdot0\cdot(0+1)(2\cdot0+4)=0$$
Both sides are equal to 0, therefore the base case $n=0$ is true.
\\
\\
Suppose the equation is true for $n = k$:
$$\sum_{i=0}^{k}i(i+1)=\dfrac{1}{6}k(k+1)(2k+4)$$
then when $n=k+1$, Left side:
$$\sum_{i=0}^{k+1}i(i+1)=\sum_{i=0}^{k}i(i+1)+(k+1)(k+1+1)$$
$$=\sum_{i=0}^{k}i(i+1)+(k+1)(k+2)$$
By induction hypothesis:
$$=\dfrac{1}{6}k(k+1)(2k+4)+(k+1)(k+2)$$
$$=\dfrac{1}{3}k(k+1)(k+2)+(k+1)(k+2)$$
$$=(k+1)(k+2)(\dfrac{1}{3}k+1)$$
Right side: $$\dfrac{1}{6}(k+1)(k+1+1)(2\cdot(k+1)+4)$$
$$=\dfrac{1}{6}(k+1)(k+1+1)(2k+6)$$
$$=(k+1)(k+2)(\dfrac{1}{3}k+1)$$
Left = Right, therefore it is true for all $n\geq0$.
\\
\\
(b).
\\
When $n=0$, Left side: $$\sum_{i=0}^{0}i2^i=0\cdot2^0= 0$$
Right side: $$(n-1)2^{n+1}+2=(0-1)\cdot2^{0+1}+2=-1\cdot2+2=0$$
Both sides are equal to 0, therefore the base case $n=0$ is true.
\\
\\
Suppose the equation is true for $n = k$:
$$\sum_{i=0}^{k}i2^i=(k-1)2^{k+1}+2$$
then when $n=k+1$, Left side:
$$\sum_{i=0}^{k+1}i2^i=\sum_{i=0}^{k}i2^i+(k+1)2^{k+1}$$
By induction hypothesis:
$$=(k-1)2^{k+1}+2+(k+1)2^{k+1}$$
$$=2^{k+1}(k-1+k+1)+2$$
$$=2^{k+1}2k+2$$
$$=2^{k+2}k+2$$
Right side: $$(k+1-1)2^{k+1+1}+2$$
$$=k2^{k+2}+2$$
$$=2^{k+2}k+2$$
Left = Right, therefore it is true for all $n\geq0$.
\\
\\
2.
\\
(a).
$$f(n)=O(r(n)) \iff \exists constant\,c_{1},n_{1}\,s.t.\,\forall n\geq n_{1}:0\leq f(n)\leq c_{1}r(n)$$
$$g(n)=O(s(n)) \iff \exists constant\,c_{2},n_{2}\,s.t.\,\forall n\geq n_{2}:0\leq g(n)\leq c_{2}s(n)$$
Let $n_{0}=max(n_{1},n_{2}):$ 
$$f(n)\cdot g(n)\iff 0\leq f(n)\cdot g(n)\leq (c_{1}c_{2})\cdot r(n)\cdot s(n),\forall n\geq n_{0}$$
Since $c_{1}$ and $c_{2}$ are constants, therefore $c_{1}c_{2}$ is a constant.
\\
By definition of Big-O, which means: $$f(n)\cdot g(n) =O(r(n)\cdot s(n))$$
Therefore the claim is true.
\\
\\
(b).
\\
If $$r(n)=n^2,\,s(n)=n$$
Then $$f(n)=O(r(n))=n^2,\,g(n)=O(s(n))=n^2$$
$$\dfrac{f(n)}{g(n)}=\dfrac{n^2}{n^2}=1$$
$$O(\dfrac{r(n)}{s(n)})=O(\dfrac{n^2}{n})=O(n)$$
Therefore the claim is not true.
\\
\\
(c).
\\
Since a positive-valued, monotonically-increasing function doesn't have to be continuous, therefore let
$$f(n)=n$$
\[
  g(n) =
  \begin{cases}
  n-1& \text{if n is odd} \\
  n & \text{if n is even} \\
  \end{cases}
\]
In this case, neither $f(n)=O(g(n))$ nor $g(n)=O(f(n))$. Therefore the claim is not true.
\\
\\
3.
\\
(a).
\\
$g(n)=O(f(n))$.
\\
Proof: Consider $c=1$ and $n_{0}=10$. Then it is true that $\forall n\geq n_{0},\, n^3\geq5n,\,\log_{2} n\geq\log_{10} n$. Since $n\geq 0, \log_{10} 5n \geq 0$. Therefore, $\forall n\geq n_{0}, 0\leq \log_{10} 5n\leq 1\cdot log_{2} n^3=c\cdot log_{2} n^3$.
By definition of Big-O, we have that $g(n)=O(f(n))$.
\\
\\
(b).
\\
$f(n)=O(g(n))$.
\\
Proof: $\log_{2} n^5=5\log_{2} n=\dfrac{5\log_{10} n}{\log_{10} 2}=\dfrac{5}{\log_{10} 2}\cdot \log_{10} n$, consider $c=1$ and $n_{0}=10^{\dfrac{5}{\log_{10} 2}}$. Then it is true that $\forall n\geq n_{0},\,\log_{10} n\cdot \log_{10} n\geq \dfrac{5}{\log_{10} 2}\cdot \log_{10} n$. Since $n\geq 0, \log_{2} n^5\geq 0$, therefore $\forall n\geq n_{0}, 0\leq \dfrac{5}{\log_{10} 2}\cdot \log_{10} n \leq 1\cdot \log_{10} n\cdot \log_{10} n \iff 0\leq \log_{2} n^5 \leq c\cdot (\log_{10} n)^2$.
By definition of Big-O, we have that $f(n)=O(g(n))$.
\\
\\
(c).
\\
$g(n)=O(f(n))$.
\\
Proof: Consider $c=1$ and $n_{0}=1$. Then it is true that $\forall n\geq n_{0},\, (\log_{2} n)^2\geq n^{\frac{1}{2}}$. Since $n\geq 0, n^{\frac{3}{2}}\geq 0$. Therefore, $\forall n\geq n_{0},\,0\leq n^{\frac{3}{2}}\leq 1\cdot n(\log_{2} n)^2=c\cdot n(\log_{2} n)^2$.
By definition of Big-O, we have that $g(n)=O(f(n))$.
\\
\\
(d).
\\
$f(n)=O(g(n))$.
\\
Proof: $n^2=2^{{\log_{2} n}^2}=2^{2\log_{2} n}$. Consider $c=1$ and $n_{0}=3$. Then it is true that $\forall n\geq n_{0},\, 2\log_{2} n\geq \sqrt n$. Since $n\geq 0$, $2^{\sqrt n}\geq 0$. Therefore, $\forall n\geq n_{0}, 0\leq 2^{\sqrt n}\leq 1\cdot 2^{2\log_{2} n}=c\cdot n^2$.
By definition of Big-O, we have that $f(n)=O(g(n))$.
\\
\\
4.
\\
$$n^{log_{2} n}={2^{log_{2} n}}^{log_{2} n}={2^{log_{2} n}}^2$$
Consider $c=1$ and $n_{0}=17$. Then it is true that $\forall n\geq n_{0},\, n\geq {\log_{2} n}^2$. Since $n\geq 0$, $2^{{\log_{2} n}^2}\geq 0$. Therefore, $\forall n\geq n_{0},\, 0\leq 2^{{\log_{2} n}^2} \leq 1\cdot 2^n \iff 0\leq n^{\log_{2} n} \leq 1\cdot 2^n = c\cdot 2^n$.
\\
By definition of Big-O, we have that $n^{\log_{2} n}=O(2^n)$.
\\
\\
5.
\\
$$\sqrt n=\log 10^{\sqrt n}$$
Consider $c=1$ and $n_{0}=1$. Then it is true that $\forall n\geq n_{0},\, 10^{\sqrt n}\geq n$. Since $n\geq 0$, $\log n\geq 0$. Therefore, $\forall n\geq n_{0}, 0\leq \log n\leq 1\cdot \log 10^{\sqrt n}\iff 0\leq \log n\leq 1\cdot \sqrt n = c\cdot \sqrt n$.
\\
By definition of Big-O, we have that $\log n=O(\sqrt n)$.
\\
\\
6.
\\
(a).
\\
$$a=4,b=2,f(n)=n,n^{\log_{b} a}=n^{\log_{2} 4}=\Theta(n^2)$$
Since $f(n)=n=O(n^{\log_{2} 4-\epsilon})$, where $\epsilon=1$, apply case 1 of the master theorem and conclude that the solution is $T(n)=\Theta(n^{\log_{b} a})=\Theta(n^{\log_{2} 4})=\Theta(n^2)$.
\\
\\
(b).
$$a=4,b=2,f(n)=n^2,n^{\log_{b} a}=n^{\log_{2} 4}=\Theta(n^2)$$
Since $f(n)=n^2=\Theta(n^{\log_{2} 4})$, apply case 2 of the master theorem and conclude that the solution is $T(n)=\Theta(n^{\log_{b} a}\lg n)=\Theta(n^{\log_{2} 4}\lg n)=\Theta(n^2\lg n)$.
\\
\\
(c).
$$a=3,b=2,f(n)=n^2,n^{\log_{b} a}=n^{\log_{2} 3}=\Theta(n^{1.5849625007})$$
Since $f(n)=n^2=\Omega(n^{\log_{2} 3+\epsilon})$, where $\epsilon=0.4150374993$, apply case 3 of the master theorem and conclude that the solution is $T(n)=\Theta(f(n))=\Theta(n^2)$.
\\
\\
(d).
$$a=81,b=3,f(n)=\dfrac{1}{3}n^4+81n^3,n^{\log_{b} a}=n^{\log_{3} 8}=\Theta(n^4)$$
Since $f(n)=O(n^4)=\Theta(n^{\log_{3} 81})$, apply case 2 of the master theorem and conclude that the solution is $T(n)=\Theta(n^{\log_{b} a}\lg n)=\Theta(n^{\log_{3} 81}\lg n)=\Theta(n^4\lg n)$.
\end{document}