\documentclass{article}
\usepackage{amsmath}
\usepackage{amssymb}
\title{CS-521-900, Assignment 3}
\author{Yiyun Zhang}
\begin{document}
\maketitle
\begin{enumerate}

\item
During exam week.

\item
\begin{verbatim}
1 int i = 0
2 function constructTree(A, bound) {
3   if(i >= n || A[i] >= bound) return null
4   root = A[i++]
5   root.left = constructTree(A, root.value)
6   root.right = constructTree(A,bound)
7   return root
8 }
9 constructTree(Array, +infinity)
\end{verbatim}
Loop invariant: $A[i]$ is less than bound. \\
Initializaation: Before the recursion starts, $A[i]$ is $A[0]$ which is the first element of the array $A$, it is also the root of the BST. The bound is +$\infty$, therefore $A[i]$ is less than bound so it is true. \\
Maintenance: For each recursion, $A[j]$'s left child is less than $A[j]$ because the bound is $A[j]$'s value, $A[j]$'s right child is larger than $A[j]$ because the bound is +$\infty$, it is also true for $j+1$, so it is true. \\
Termination: When the recursion stops, $A[n]$ is the last element of the array $A$, which is also the last element of the BST. The bound is +$\infty$, therefore $A[n]$ is less than bound so it is true. \\
Running time: For function $contructTree()$, the recurrance relation is $T(n) = 2T(n) + c$, therefore the running time complexity is $O(n)$.

\item
\begin{verbatim}
1 function constructTree(A, i) {
2   if(i == 0) return null
3   root = A[i/2]
4   root.left = constructTree(A, i/2)
5   root.right = constructTree(A, 3i/2)
6   return root
7 }
8 constructTree(Array, n)
\end{verbatim}
Loop invariant: $A[i]$ is larger than its left child and smaller than its right child. \\
Initializaation: Before the recursion starts, $A[i]$ is $A[n]$ which is the last element of the array $A$. Since there is no BST yet, so the statement is true. \\
Maintenance: For each recursion, $A[j]$'s left child is the $A[j/2]$ and the right child is $A[3j/2]$, since the array $A$ is a sorted array therfore $A[j]$ is always larger than $A[j/2]$ and smaller than $A[3j/2]$. It is also true for $j+1$, so it is true. \\
Termination: When the recursion stops, $A[0]$ is the first element of the array $A$. Since it is the smallest number in the array and it has no children, therfore the statement is true. \\
Running time: For function $contructTree()$, the recurrance relation is $T(n) = 2T(n/2) + c$, therefore the running time complexity is $O(n)$. \\

\item
For a directed graph $G = (V, E)$, computing out-degrees and in-degrees of every vertex both take $O(|E|+|V|)$ time, because we have to go through all adjacency lists to count the arcs incident from or directed away.

\item
Based on $Theorem 22.10$: In a depth-first search of an undirected graph G, every edge of G is either a tree edge or a back edge. Therefore, if there is no back edge, there is only tree edge which means the undirected graph is acyclic. Thus, if there is a back edge exists, there is a circle contains in the given undirected graph G.
\begin{verbatim}
1  function hasCycle(G) {
2    for each vertex u in G.V
3      u.color = WHITE
4      visited = NIL
5    time = 0
6    for each vertex u in G.V
7      if u.color == WHITE
8        if hasCycle-VISIT(G, u)
9          return true
10    return false
11  }
12  function hasCycle-VISIT(G, u) {
13    time = time + 1
14    u.d = time
15    u.color = GRAY
16    for each v in G.Adj[u]
17      if (u, v) not in visited
18        visited.add((u, v))
19        visited.add((v, u))
20        if v.color == GRAY
21          return true
22        if v.color == WHITE
23          if hasCycle-VISIT(G, v)
24            return true
25    u.color = BLACK
26    time = time + 1
27    u.f = time
28    return false
29  }
30  hasCycle(G)
\end{verbatim}
According to the book: GRAY: indicates a back edge. $(u,v)$ and $(v,u)$ are really the same edge in an undirected graph. \\
Loop invariant: unvisited vertex's color is white
Initializaation: Before the recursion starts, all vertex has no color so the statement is true.
Maintenance: For each recursion, current vertex's color is set to black, it is also true for next recursion. So the statement is true.
Termination: When the recursion stops, all vertex has been visited, if there is back edge exists, those vertex are grey, others are black.
Running time: If there is no back edge, there will be maximum $|V|-1$ distinct edges in the graph, therefore it is the upper bound before a back edge is found. Thus, the running time is $O(|V|)$

\end{enumerate}
\end{document}