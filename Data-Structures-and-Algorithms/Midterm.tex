\documentclass{article}
\usepackage{amsmath}
\usepackage{amssymb}
\title{CS-521-900, Midterm}
\author{Yiyun Zhang}
\begin{document}
\maketitle

\begin{enumerate}
\item 
{\bf I)} $a=4, b=4, n^{\log_b a}=n, f(n)=4n+4, f(n)=O(n)$ Therefore $f(n)=n^{\log_b a}$, thus case 2. $T(n)=\theta (n^{\log_b a\log n)}=\theta(n\log n)$, which is $c.\theta(n\log n)$.

{\bf II)} Since $T(n/3) \geq T(n/4)$, Therefore we can rewrite $T(n)$ to $T(n) \leq 3T(n/3)+n$ In this case, $a=3, b=3, n^{\log_b a}=n, f(n)=n$, thus case 2.  $T(n)=\theta (n^{\log_b a\log n)}=\theta(n\log n)$, which is $c.\theta(n\log n)$.

{\bf III)} $T(n)=T(n-2)+n^2+n=T(n-4)+n^2+n^2+n+n=T(n-6)+n^2+n^2+n^2+n+n+n=...=T(\dfrac{n}{2}(n-2)+\dfrac{n}{2}n^2+\dfrac{n}{2}n)=\dfrac{n^3}{2}+\dfrac{n^2-2n}{2}+\dfrac{n^2}{2}=O(n^3)$Therefore, it is $e.\theta(n^3)$.

{\bf III)} $T(0)=1,T(3)=5,T(6)=5^2,T(9)=5^3,T(12)=5^4...T(n)=5^{n/3}$ Therefore, it is $g.\theta(5^{n/3})$.

\item
{\bf I)}
\begin{verbatim}
1 Select(A,p,q,i){
2   Divide A to n/5 groups of size 5
3   Find the median of each group of 5 by brute force
      and store them in a set A'of size n/5
4    Use  Select(A',p,q,i) to find the median x of n/5 medians
5    Partition the n elements around x. Let k=q-p+1(rank of x)
6    if (i==k):
7      return A[n/3] to A[2n/3] 
8    if(i>n/3):
9      Select(A,n/3,q,(i-n/3))
10   if(i<2n/3):
11     Select(A,2n/3,q,i)
12 }
\end{verbatim}
Similar to the original $Select$ Algorithm, since we only need n/3 of the elements between n/3 to 2n/3, so we don't have to sort the elements smaller then the pivot n/3 or the elements larger than pivot 2n/3. Therefore the complexity is $T(n)=T(n/5)+T(3n/4)+\theta(1/3n)=\theta(n)$

{\bf II)}
\begin{verbatim}
1 Select(A,p,q,i){
2   Divide A to n/5 groups of size 5
3   Find the median of each group of 5 by brute force
      and store them in a set A'of size n/5
4    Use  Select(A',p,q,i) to find the median x of n/5 medians
5    Partition the n elements around x. Let k=q-p+1(rank of x)
6    if (i==k):
7      return A[n/3] to A[2n/3] 
8    if(i>n/3):
9      Select(A,n/3,q,(i-n/3))
10   if(i<2n/3):
11     Select(A,2n/3,q,i)
12 }
13 Quicksort(res,p,r)
\end{verbatim}
Since the size of $res$ is $n/3$, therefore the overall complexity is $\theta(n+n/3\log n/3)$.

\item
\begin{verbatim}
1 Let y be the person number
2 Select(A,p,q,i,y){
3   Divide A to n/5 groups of size 5
4   Find the median of each group of 5 by brute force
      and store them in a set A'of size n/5
5    Use  Select(A',p,q,i) to find the median x of n/5 medians
6    Partition the n elements around x. Let k=q-p+1(rank of x)
7    if (i==y):
8      return x
9    if(i>y):
10      Select(A,p,k,i,y+1)
11   if(i<y):
12     Select(A,k.q.i-k,y-1)
13 }
\end{verbatim}
The overall complexity is $\theta(\log n)$.

\item
{\bf I)}
\begin{verbatim}
1 A = combine(A,B), n = 2m
2 Select(A,p,q,i){
3   Divide A to n/5 groups of size 5
4   Find the median of each group of 5 by brute force
      and store them in a set A'of size n/5
5    Use  Select(A',p,q,i) to find the median x of n/5 medians
6    Partition the n elements around x. Let k=q-p+1(rank of x)
7    if (i==k):
8      return x 
9    if(i<k):
10      Select(A,p,k-1,i)
11   else:
12     Select(A,k,q,i-k)
13 }
\end{verbatim}
Combine $A$ and $B$ to create the union, it takes constant time. Then use $Select()$ to find the median of the union. The complexity of the $Select()$ is $\theta(n)$. Therefore the overall complexity is $\theta(n)$. 


{\bf II)}
\begin{verbatim}
1 Make A a min heap
2 Make B a min heap
3 Let newRoot = -infinity
4 For each key level:
5  newHeap.add(A.currentLevelElements)
6  newHeap.add(B.currentLevelElements)
7 End
8 Heap-Extract-Max(newHeap)
9 Find the median x of middle key level by brute force
10 Return x
\end{verbatim}
Since $A$ and $B$ are sorted array, it takes constant time to build new heap(simply copy paste). $A$ and $B$ has the same size, therefore the size of the new heap is $2n$. Find the median x in the middle level by brute force take constant time. The $Heap-Extract-Max(newHeap)$ function calls $Heapify()$ which takes $O(\log n)$ time, therefore the overall complexity is $\theta(\log n)$

\item
\item

\begin{table}[h!tb]
\Large
\centering
{\begin{tabular}{ |c|c|c|c|c|c|c|c|c| }
  \hline
   & $A$ & $B$ & $O$ & $o$ & $\Omega$ & $\omega$ & $\Theta$ \\
  \hline
  a. & $lg^kn$ & $n^\epsilon$  & True & True & False & False & False\\ 
  \hline
  b. & $n^k$ & $C^n$  & True & True & False & False & False\\ 
  \hline
  c.  & $\sqrt{n}$ & $n^{sin(n)}$  & False & False & False & False & False\\ 
  \hline
  d. & $2^n$ & $2^{n/2}$  & False &False & True & True & False\\ 
  \hline
  e. & $n^{lgc}$ & $c^{lgn}$  & True &False & True & False & True\\ 
  \hline
  f. & $lg(n!)$ & $lg(n^n)$  & True & False & True & False & True\\
 

  \hline
\end{tabular}}
\label{}
\end{table}

\end{enumerate}
\end{document}